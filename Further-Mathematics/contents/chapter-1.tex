\section{映射与函数}
\label{sec:ch-1-sec-1}
1. 求下列函数的自然定义域:
\begin{shrinkeq}{-2.5ex}
    \begin{align*}
      \tag 1
      y = \frac{1}{ 1-x^2},\ \mathbb{D}(x)\ne\pm 1
    \end{align*}

    \begin{align*}
      \tag 2
      y = \sqrt{ 3x+2 },\ \mathbb{D}(x)\in[-\frac{2}{3},+\infty)
    \end{align*}

    \begin{align*}
      \tag 3
      y = \frac{1}{x}-\sqrt{1-x^2},\ \mathbb{D}(x)\ne\pm 1, 0
    \end{align*}

    \begin{align*}
      \tag 4
      y = \frac{1}{ \sqrt{ 4-x^2 } },\ \mathbb{D}(x)\ne\pm 1
    \end{align*}

    \begin{align*}
      \tag 5
      y = \sin{\sqrt{x}},\ \mathbb{D}(x)\in[0,+\infty)
    \end{align*}

    \begin{align*}
      \tag 6
      y = \tan(x + 1),\ \mathbb{D}(x)\ne k\pi+\frac{\pi}{2}-1,k\in\mathbb{Z}
    \end{align*}

    \begin{align*}
      \tag 7
      y = \arcsin(x-3)
    \end{align*}

    \begin{align*}
      \tag 8
      y = \sqrt{ 3-x }+\arcsin{\frac{1}{x}}
    \end{align*}

    \begin{align*}
      \tag 9
      y = \ln{ x+1 }
    \end{align*}

    \begin{align*}
      \tag {10}
      y = e^{\frac{1}{x}}
    \end{align*}
\end{shrinkeq}

\begin{equation}
  \label{}
\ne \neq k\pi
\end{equation}


% \begin{align}
%   \label{align:1}
%   % 使用 nonumber 取消该行的序号
%   y & = \sqrt{ 3x+2 }\\
%     & = (3x+2)^{ \frac{ 1 }{ 2 }} \nonumber
% \end{align}
% \begin{align}
%   \label{align:2}
%   % 使用 nonumber 取消该行的序号
%   y & = \sqrt{ 3x+2 }\\
%     & = (3x+2)^{ \frac{ 1 }{ 3 }}\\
%     & = (3x+2)^{ \frac{ 1 }{ 2 }} \nonumber
% \end{align}

% \begin{align}
%   \label{align:3}
%   x^2+y^2&=1 &
%                x^3+y^3&=1 \\
%   x&=\sqrt{1-y^2}&
%                    x&=\sqrt[3]{1-y^3}
% \end{align}


%%% Local Variables:
%%% mode: latex
%%% TeX-master: "../further-mathematics"
%%% End:
