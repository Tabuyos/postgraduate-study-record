\documentclass[a4paper,zihao=-4,UTF8]{ctexbook}
% \documentclass[a4paper,UTF8]{article}
\usepackage[left=3.17cm,right=3.17cm,top=2.54cm,bottom=2.54cm]{geometry}
\usepackage{booktabs,bigstrut,multirow}
\usepackage{enumitem}
\usepackage[warnings-off={mathtools-colon}]{unicode-math}
\usepackage{caption,subcaption}
\captionsetup{font=small,skip=6pt,textfont=it}
\renewcommand{\figurename}{{\kaishu 图}}
\renewcommand{\tablename}{{\kaishu 表}}
\usepackage[breaklinks,colorlinks,bookmarksnumbered]{hyperref}
\usepackage{zhlineskip}
\usepackage{zhlipsum}
\setCJKmainfont{Noto Sans CJK SC Light}
\renewcommand{\emph}{\kaishu}
\usepackage{gbt7714}
\title{\heiti 高等数学}
\author{\itshape{Tabuyos}\thanks{Aaron Liew}}
\date{\small{\today}}
\begin{document}
\maketitle
\pagenumbering{Roman}
\tableofcontents
\newpage
\pagenumbering{arabic}
% 只能放在正文, 并且会另起一页
% \section{映射与函数}
\label{sec:ch-1-sec-1}
1. 求下列函数的自然定义域:
\begin{shrinkeq}{-2.5ex}
    \begin{align*}
      \tag 1
      y = \frac{1}{ 1-x^2}
    \end{align*}

    \begin{align*}
      \tag 2
      y = \sqrt{ 3x+2 }
    \end{align*}

    \begin{align*}
      \tag 3
      y = \frac{1}{x}-\sqrt{1-x^2}
    \end{align*}

    \begin{align*}
      \tag 4
      y = \frac{1}{ \sqrt{ 4-x^2 } }
    \end{align*}

    \begin{align*}
      \tag 5
      y = \sin{\sqrt{x}}
    \end{align*}

    \begin{align*}
      \tag 6
      y = \tan(x + 1)
    \end{align*}

    \begin{align*}
      \tag 7
      y = \arcsin(x-3)
    \end{align*}

    \begin{align*}
      \tag 8
      y = \sqrt{ 3-x }+\arcsin{\frac{1}{x}}
    \end{align*}

    \begin{align*}
      \tag 9
      y = \ln{ x+1 }
    \end{align*}

    \begin{align*}
      \tag {10}
      y = e^{\frac{1}{x}}
    \end{align*}
\end{shrinkeq}

% \begin{align}
%   \label{align:1}
%   % 使用 nonumber 取消该行的序号
%   y & = \sqrt{ 3x+2 }\\
%     & = (3x+2)^{ \frac{ 1 }{ 2 }} \nonumber
% \end{align}
% \begin{align}
%   \label{align:2}
%   % 使用 nonumber 取消该行的序号
%   y & = \sqrt{ 3x+2 }\\
%     & = (3x+2)^{ \frac{ 1 }{ 3 }}\\
%     & = (3x+2)^{ \frac{ 1 }{ 2 }} \nonumber
% \end{align}

% \begin{align}
%   \label{align:3}
%   x^2+y^2&=1 &
%                x^3+y^3&=1 \\
%   x&=\sqrt{1-y^2}&
%                    x&=\sqrt[3]{1-y^3}
% \end{align}


%%% Local Variables:
%%% mode: latex
%%% TeX-master: "../further-mathematics"
%%% End:

% 可以放在正文和导言区, 并且不会另起一页
% \section{映射与函数}
\label{sec:ch-1-sec-1}
1. 求下列函数的自然定义域:
\begin{shrinkeq}{-2.5ex}
    \begin{align*}
      \tag 1
      y = \frac{1}{ 1-x^2}
    \end{align*}

    \begin{align*}
      \tag 2
      y = \sqrt{ 3x+2 }
    \end{align*}

    \begin{align*}
      \tag 3
      y = \frac{1}{x}-\sqrt{1-x^2}
    \end{align*}

    \begin{align*}
      \tag 4
      y = \frac{1}{ \sqrt{ 4-x^2 } }
    \end{align*}

    \begin{align*}
      \tag 5
      y = \sin{\sqrt{x}}
    \end{align*}

    \begin{align*}
      \tag 6
      y = \tan(x + 1)
    \end{align*}

    \begin{align*}
      \tag 7
      y = \arcsin(x-3)
    \end{align*}

    \begin{align*}
      \tag 8
      y = \sqrt{ 3-x }+\arcsin{\frac{1}{x}}
    \end{align*}

    \begin{align*}
      \tag 9
      y = \ln{ x+1 }
    \end{align*}

    \begin{align*}
      \tag {10}
      y = e^{\frac{1}{x}}
    \end{align*}
\end{shrinkeq}

% \begin{align}
%   \label{align:1}
%   % 使用 nonumber 取消该行的序号
%   y & = \sqrt{ 3x+2 }\\
%     & = (3x+2)^{ \frac{ 1 }{ 2 }} \nonumber
% \end{align}
% \begin{align}
%   \label{align:2}
%   % 使用 nonumber 取消该行的序号
%   y & = \sqrt{ 3x+2 }\\
%     & = (3x+2)^{ \frac{ 1 }{ 3 }}\\
%     & = (3x+2)^{ \frac{ 1 }{ 2 }} \nonumber
% \end{align}

% \begin{align}
%   \label{align:3}
%   x^2+y^2&=1 &
%                x^3+y^3&=1 \\
%   x&=\sqrt{1-y^2}&
%                    x&=\sqrt[3]{1-y^3}
% \end{align}


%%% Local Variables:
%%% mode: latex
%%% TeX-master: "../further-mathematics"
%%% End:

\chapter{Title}
\begin{equation}
  \label{eq:1}
  a+b=1
\end{equation}
\begin{table}[htbp]
  \centering
  \caption{不同激励对应的特解}
  \small
  \begin{tabular}{ccc}
    \toprule
    激励 $f(x)$   & \multicolumn{2}{c}{特解 $y_p(x)$}  \\
    \midrule
    \multirow{2}*{\mbox{$x^m$}} &  $P_mx^m+P_{m-1}x^{m-1}+\cdots+P_1x+P_0$   & 所有特征根不相等\\
                  &  $x^r\left(P_mx^m+P_{m-1}x^{m-1}+\cdots+P_1x+P_0\right)$              & 有$r$重相等特征根\\
    \midrule
    \multirow{3}*{$e^{\alpha x}$} & $Pe^{\alpha x}$                     & $\alpha$  不等于特征根\\
                  & $(P_1x+P_0)e^{\alpha x}$                                    & $\alpha$等于特征单根 \\
                  & $P_rx^r+P_{r_1}x^{r_1}+\cdots+P_1x+P_0$                     & $\alpha$等于$r$重特征单根 \\
    \midrule
    $\cos(\beta x)$或$\sin(\beta x)$ & $ P\cos\beta x+Q\sin \beta x$   & 所有特征根不相等 \\
    \bottomrule
  \end{tabular}%
  % \label{tab:addlabel}%
\end{table}%
\end{document}

%%% Local Variables:
%%% mode: latex
%%% TeX-master: t
%%% End:
